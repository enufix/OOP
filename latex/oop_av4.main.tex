%%%%%%%%%%%%%%%%%%%%%%%%%%%%%%%%%%%%%%%%%
%%%%%%%%%% Content starts here %%%%%%%%%%
%%%%%%%%%%%%%%%%%%%%%%%%%%%%%%%%%%%%%%%%%
\section{Динамичка алокација на меморија}

\begin{frame}{Задача 1}{Класа за динамичко поле}
Да се напише класа за работа со еднодимензионални полиња од целобројни елементи.
За полето се чуваат информации за вкупниот капцитет на полето, тековниот број на
елементи. Резервацијата на меморијата да се врши динамички. 
Да се преоптоварат следните оператори 
\begin{itemize}
  \item \texttt{[]} за пристап до елемент и промена на вредноста на
  елемент од полето
  \item \texttt{+=} за додавање нови броеви во полето и притоа ако е исполнет
  капацитетот на полето да се зголеми за 100\%.
\end{itemize}
Да се напише главна програма каде ќе се инстанцира објект од класата и во него
ќе се внесат N броеви од тастатура и потоа да се испечатат елементите на полето,
неговиот капацитет и вкупниот број на елементи.
\end{frame}

\begin{frame}[fragile]{Задача 1}{Решение 1/3}
\lstinputlisting[lastline=22]{src/av4/z1.cpp}
\end{frame}

\begin{frame}[fragile]{Задача 1}{Решение 2/3}
\lstinputlisting[firstline=23,lastline=52]{src/av4/z1.cpp}
\end{frame}

\begin{frame}[fragile]{Задача 1}{Решение 3/3}
\lstinputlisting[firstline=52]{src/av4/z1.cpp}
\end{frame}

\section{Композиција}

\begin{frame}{Задача 2}{Композиција}
Да се напише класа за работа со веб сервери (\texttt{WebServer}). За секој веб
сервер се чува:
\begin{itemize}
  \item неговото име (max 30 знаци)
  \item листа од веб страници (динамичка низа објекти од класата
  \texttt{WebPage}).
\end{itemize}
За секоја веб страница се чува:
\begin{itemize}
  \item url (max 100 знаци)
  \item содржина (динмички алоцирана низа од знаци). 
\end{itemize}
За класата \texttt{WebServer} да се преотоварат операторите:
\begin{itemize}
  \item \texttt{+=} за додавање нова веб страница во серверот
  \item \texttt{-=} за бришење на веб страница од веб серверот. 
\end{itemize}

\end{frame}

\begin{frame}[fragile]{Задача 2}{Решение 1/5}
\lstinputlisting[lastline=22]{src/av4/z2.cpp}
\end{frame}

\begin{frame}[fragile]{Задача 2}{Решение 2/5}
\lstinputlisting[firstline=23,lastline=39]{src/av4/z2.cpp}
\end{frame}

\begin{frame}[fragile,shrink=10]{Задача 2}{Решение 3/5}
\lstinputlisting[firstline=40,lastline=70]{src/av4/z2.cpp}
\end{frame}

\begin{frame}[fragile]{Задача 2}{Решение 4/5}
\lstinputlisting[firstline=71,lastline=100]{src/av4/z2.cpp}
\end{frame}

\begin{frame}[fragile]{Задача 2}{Решение 5/5}
\lstinputlisting[firstline=101]{src/av4/z2.cpp}
\end{frame}

\begin{frame}{Задача 3}{Композиција}
Да се напише класа \texttt{Datum}, во која ќе се чуваат ден, месец и година (цели
броеви).\\
Да се напише класа \texttt{Vraboten}, за кој се чува име (не повеќе од 100
знаци) и датум на раѓање (објект од Datum).\\ 
Да се напише класа Firma, во која се чува име на фирмата (не повеќе од 100
знаци) и низа од вработени (динамичи алоцирана низа од објекти од Vraboten).
Да се преоптовари операторот += за додавање на вработен во фирмата.
За оваа класа да се имплементираат метод кој ќе ги печати
сите вработени во фирмата и метод кој ќе го пронајде и испечати најмладиот вработен.
\end{frame}

\begin{frame}[fragile]{Задача 3}{Решение 1/5}
\lstinputlisting[lastline=25]{src/av4/z3.cpp}
\end{frame}

\begin{frame}[fragile]{Задача 3}{Решение 2/5}
\lstinputlisting[firstline=27,lastline=45]{src/av4/z3.cpp}
\end{frame}

\begin{frame}[fragile]{Задача 3}{Решение 3/5}
\lstinputlisting[firstline=46,lastline=75]{src/av4/z3.cpp}
\end{frame}

\begin{frame}[fragile]{Задача 3}{Решение 4/5}
\lstinputlisting[firstline=76,lastline=105]{src/av4/z3.cpp}
\end{frame}

\begin{frame}[fragile]{Задача 3}{Решение 5/5}
\lstinputlisting[firstline=106]{src/av4/z3.cpp}
\end{frame}

\section{Наследување}

\begin{frame}[fragile]{Наследување}
\begin{lstlisting}
class ime_izvedena : pristap ime_osnovna {
    // definicija i implementacija na klasata
};
\end{lstlisting}
\begin{exampleblock}{Пример наследување со \texttt{public} пристап}
\begin{lstlisting}
class Osnovna {
private:
    int a;
public:
    Osnovna(int a) { this->a = a; }
    Osnovna() {}
}
class Izvedena : public Osnovna {
private:
    int b;
public:
    Izvedena(int a, int b) : Osnovna(a) {
        this->b = b;
    }
}
\end{lstlisting}
\end{exampleblock}
\end{frame}

\begin{frame}{Задача 4}{Наследување}
Да се напише класа за точка во дводимензионален (2Д) координатен систем. Од оваа
класа да се изведе класа за точка во тродимензионален (3Д) кооридинатен систем.
\end{frame}

\begin{frame}[fragile]{Задача 4}{Решение}
\lstinputlisting[lastline=25]{src/av4/z4.cpp}
\end{frame}

\begin{frame}{Задача 5}{Наследување}
Да се напише класа која ќе опишува банкарска сметка. За сметката се чуваат
информации за бројот, името на сопственикот, како и салдото (моменталната
состојба). За оваа класа да се имплементираат методи за додавање и повлекување
средства од сметката, како метод кој пачати на екран преглед на сметката.
Доколку сопственикот нема доволно средства при повлекувањето се печати соодветна порака.\\
Од оваа класа да се изведе класа за банкарска сметка со дозволен максимален
заем на кој се пресметува соодветна камата. За оваа класа да се преоптоварат
методите за повлекување средства и печатање преглед на сметката.
\end{frame}

\begin{frame}[fragile]{Задача 5}{Решение 1/4}
\lstinputlisting[lastline=15]{src/av4/z5.cpp}
\end{frame}

\begin{frame}[fragile]{Задача 5}{Решение 2/3}
\lstinputlisting[firstline=16,lastline=34]{src/av4/z5.cpp}
\end{frame}

\begin{frame}[fragile]{Задача 5}{Решение 3/4}
\lstinputlisting[firstline=35,lastline=63]{src/av4/z5.cpp}
\end{frame}

\begin{frame}[fragile]{Задача 5}{Решение 4/4}
\lstinputlisting[firstline=64]{src/av4/z5.cpp}
\end{frame}
