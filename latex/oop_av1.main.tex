%%%%%%%%%%%%%%%%%%%%%%%%%%%%%%%%%%%%%%%%%
%%%%%%%%%% Content starts here %%%%%%%%%%
%%%%%%%%%%%%%%%%%%%%%%%%%%%%%%%%%%%%%%%%%

\begin{frame}{Задачa 1}
Да се напише програма која ќе споредува два датуми (ден, месец, година) и ќе ја
пресмета разликата во денови од едниот до другиот датум. Пресметките да се
реализираат во посебни функции.\\
\emph{За датумот да се дефинира посебна структура} \texttt{datum}.
\end{frame}

\begin{frame}[fragile]{Задача 1}{Решение 1/3}
\lstinputlisting[lastline=9]{src/av1/z1.c}
\end{frame}

\begin{frame}[fragile]{Задача 1}{Решение 2/3}
\lstinputlisting[firstline=11,lastline=30]{src/av1/z1.c}
\end{frame}

\begin{frame}[fragile]{Задача 1}{Решение 3/3}
\lstinputlisting[firstline=32]{src/av1/z1.c}
\end{frame}

\begin{frame}{Задачa 2}
Да се напише програма која ќе го пресметува векторскиот и скаларниот производ на
два вектори. Векторите се претставени со координати во тродимензионален
координатен систем. Скаларниот и векторскиот производ да се пресметуваат со
посебни функции.
\emph{За вектор да се дефинира посебна структура} \texttt{вектор}.
\end{frame}

\begin{frame}[fragile]{Задача 2}{Решение 1/2}
\lstinputlisting[lastline=21]{src/av1/z2.c}
\end{frame}

\begin{frame}[fragile]{Задача 2}{Решение 2/2}
\lstinputlisting[firstline=23]{src/av1/z2.c}
\end{frame}

\begin{frame}{Задачa 3}
Да се напише програма во која ќе се дефинира структура за претставување
комплексни броеви. Потоа да се напишат функции за собирање, одземање и множење
на два комплексни броеви. Програмата да се тестира во главна програма во која се
вчитуваат два комплексни броја од стандарден влез.
\end{frame}

\begin{frame}[fragile]{Задача 3}{Решение 1/3}
\lstinputlisting[lastline=20]{src/av1/z3.c}
\end{frame}

\begin{frame}[fragile]{Задача 3}{Решение 2/3}
\lstinputlisting[firstline=22,lastline=33]{src/av1/z3.c}
\end{frame}

\begin{frame}[fragile]{Задача 3}{Решение 3/3}
\lstinputlisting[firstline=35]{src/av1/z3.c}
\end{frame}

\begin{frame}{Задачa 4}
Од стандарден влез се читаат непознат број податоци за студенти. Податоците се
внесуваат така што во секој ред се дава:
\begin{itemize}
  \item името
  \item презимето
  \item бројот на индекс (формат xxyzzzz)
  \item четири броја (поени од секоја задача)
\end{itemize}         
     со произволен број празни места или табулатори меѓу нив.\\
Да се напише програма која ќе испечати список на студенти, каде во секој ред ќе
има: презиме, име, број на индекс, вкупен број на бодови сортиран според бројот
на бодови. При тоа имињата и презимињата да се напишат со голема почетна буква.
\end{frame}

\begin{frame}[fragile]{Задача 4}{Решение 1/3}
\lstinputlisting[lastline=16]{src/av1/z4.c}
\end{frame}

\begin{frame}[fragile]{Задача 4}{Решение 2/3}
\lstinputlisting[firstline=18,lastline=28]{src/av1/z4.c}
\end{frame}

\begin{frame}[fragile]{Задача 4}{Решение 3/3}
\lstinputlisting[firstline=30]{src/av1/z4.c}
\end{frame}


\begin{frame}{Задачa 5}
Да се напише програма која од стандарден влез ќе чита податоци за држави и на
екран ќе го отпечати името и презимето на претседателот на државата чиј што главен град
има најмногу жители.
\begin{itemize}
  \item Податоци за држава: име, претседател, главен град и број на жители
  \item Податоци за град: име и број на жители
  \item Податоци за претседател: име, презиме и политичка партија
\end{itemize}
\end{frame}

\begin{frame}[fragile]{Задача 5}{Решение 1/3}
\lstinputlisting[lastline=19]{src/av1/z5.c}
\end{frame}

\begin{frame}[fragile]{Задача 5}{Решение 2/3}
\lstinputlisting[firstline=21,lastline=42]{src/av1/z5.c}
\end{frame}

\begin{frame}[fragile]{Задача 5}{Решение 3/3}
\lstinputlisting[firstline=43]{src/av1/z5.c}
\end{frame}

\begin{frame}{За дома}
Задача 4 да се модифицира така што во структурата со која е опишан студентот да
се додаде низа од предмети (не повеќе од 10) со оценки од испит. Да се напише
програма која ќе отпечати список со студенти сортирани според нивниот просек,
почнувајќи од студентот со најголем просек. Податоците за предметот се: име на
предметот и оценка по предметот.
\end{frame}
