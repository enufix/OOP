%%%%%%%%%%%%%%%%%%%%%%%%%%%%%%%%%%%%%%%%%
%%%%%%%%%% Content starts here %%%%%%%%%%
%%%%%%%%%%%%%%%%%%%%%%%%%%%%%%%%%%%%%%%%%

\begin{frame}{Problem 1}

Create a class hierarchy for \texttt{JetCar} which derives from two classes, car and jet
(diamond problem).

\end{frame}

\begin{frame}[fragile]{Problem 1}{Solution 1/4}
\lstinputlisting[lastline=22]{src/av8/p81.cpp}
\end{frame}

\begin{frame}[fragile]{Problem 1}{Solution 2/4}
\lstinputlisting[firstline=24,lastline=40]{src/av8/p81.cpp}
\end{frame}

\begin{frame}[fragile]{Problem 1}{Solution 3/4}
\lstinputlisting[firstline=41,lastline=67]{src/av8/p81.cpp}
\end{frame}

\begin{frame}[fragile]{Problem 1}{Solution 4/4}
\lstinputlisting[firstline=69]{src/av8/p81.cpp}
\end{frame}

\begin{frame}{Problem 2}
\begin{scriptsize}
Implement class for \texttt{Product} that keeps name and price. Then implement
abstract class \texttt{Discount} that have two pure virtual functions for price
and price on discount. From this classes derive:

\begin{itemize}
  \item \texttt{FoodProduct} - additionally keeps calories
  \item \texttt{DigitalProduct} - additionally keeps size (in MB).
\end{itemize}

Implement global function \texttt{total\_discount} that will compute the total
discount of N products passed as argument to this function.


\end{scriptsize}
 
\end{frame}

\begin{frame}[fragile]{Problem 2}{Solution 1/4}
\lstinputlisting[lastline=25]{src/av8/p82.cpp}
\end{frame}

\begin{frame}[fragile]{Problem 2}{Solution 2/4}
\lstinputlisting[firstline=27,lastline=44]{src/av8/p82.cpp}
\end{frame}

\begin{frame}[fragile]{Problem 2}{Solution 3/4}
\lstinputlisting[firstline=45,lastline=61]{src/av8/p82.cpp}
\end{frame}

\begin{frame}[fragile]{Problem 2}{Solution 4/4}
\lstinputlisting[firstline=63]{src/av8/p82.cpp}
\end{frame}
