%%%%%%%%%%%%%%%%%%%%%%%%%%%%%%%%%%%%%%%%%
%%%%%%%%%% Content starts here %%%%%%%%%%
%%%%%%%%%%%%%%%%%%%%%%%%%%%%%%%%%%%%%%%%%

\begin{frame}[fragile]{Дефинирање класа и објект}
\begin{exampleblock}{Дефиниција на класа}
\begin{lstlisting}
class Ime_na_klasata {
private:
    /* deklaracija na promenlivi i metodi koi ne se vidlivi nadvor od klasata */
public:
    /* deklaracija na promenlivi i metodi koi se vidlivi nadvor od klasata */
};
\end{lstlisting}
\end{exampleblock}
\begin{exampleblock}{Инстанцирање објект}
\begin{lstlisting}
Ime_na_klasata ime_na_objektot;
\end{lstlisting}
\end{exampleblock}
\end{frame}

\begin{frame}[fragile,shrink=10]{Дефинирање класа}{Пример}
\lstinputlisting{src/av3/ex1.cpp}
\end{frame}

\begin{frame}{Животен циклус на објектите}
\begin{enumerate}
  \item Конструктор (Constructor) - посебен метод на секоја класа кој се
  повикува секогаш кога се креира (инстанцира) објект од класата
  \item Деструктор (Destructor) - посебен метод на секоја класа кој се
  повикува кога треба да се ослободи меморијата која ја опфатил конструираниот
  објект
\end{enumerate}
\end{frame}

\begin{frame}{Задача 1}{Класа за триаголник}
Да се напише класа за опишување на геометриско тело триаголник. Во класата да се
напишат методи за пресметување на плоштината и периметарот на триаголникот.\\
Потоа да се напише главна програма во која ќе се инстнацира еден објект од оваа
класа со вредности за страните кои претходно ќе се прочитаат од стандарден влез.
На овој објект да се повикат соодветните методи за пресметување на плоштината и
периметарот.
\end{frame}

\begin{frame}[fragile]{Задача 1}{Решение 1/2 (Класа)}
\lstinputlisting[lastline=27]{src/av3/z1.cpp}
\end{frame}

\begin{frame}[fragile]{Задача 1}{Решение 2/2 (main)}
\lstinputlisting[firstline=28]{src/av3/z1.cpp}
\end{frame}

\begin{frame}{Задача 2}{Класа за вработен}
Да се напише класа во која ќе се чуваат основни податоци за вработен:
\begin{itemize}
  \item име
  \item плата
  \item работна позиција (работната позиција може да биде вработен, директор
или шеф)
\end{itemize}
Напишете главна програма во која се читаат од стандарден влез податоци за N
вработени, а потоа се пачати листа на вработените сортирани според висината на
платата во опаѓачки редослед.
\end{frame}

\begin{frame}[fragile]{Задача 2}{Решение 1/4}
\lstinputlisting[lastline=26]{src/av3/z2.cpp}
\end{frame}

\begin{frame}[fragile]{Задача 2}{Решение 2/4}
\lstinputlisting[firstline=28,lastline=46]{src/av3/z2.cpp}
\end{frame}

\begin{frame}[fragile]{Задача 2}{Решение 3/4}
\lstinputlisting[firstline=48,lastline=58]{src/av3/z2.cpp}
\end{frame}

\begin{frame}[fragile]{Задача 2}{Решение 4/4}
\lstinputlisting[firstline=60]{src/av3/z2.cpp}
\end{frame}

\begin{frame}{Задача 3}{Класа за e-mail}
Да се напише класа која опишува една e-mail порака. Во класата треба да се
имплементира метод за прикажување на целокупната порака на екран.\\
Потоа да се напише главна програма во која се внесуваат параметрите на пораката,
се инстанцира објект од оваа класа и се печати на екран неговата содржина. За
проверување на валидноста на e-mail пораката (постоење на знакот @ во адресата)
да се напише соодветна функција.
\end{frame}

\begin{frame}[fragile]{Задача 3}{Решение 1/4}
\lstinputlisting[lastline=24]{src/av3/z3.cpp}
\end{frame}

\begin{frame}[fragile]{Задача 3}{Решение 2/3}
\lstinputlisting[firstline=25,lastline=52]{src/av3/z3.cpp}
\end{frame}

\begin{frame}[fragile]{Задача 3}{Решение 3/3}
\lstinputlisting[firstline=53]{src/av3/z3.cpp}
\end{frame}

% \begin{frame}[fragile]{Задача 4}{Класа за комплексни броеви}
% \lstinputlisting[lastline=21]{src/av3/z4.cpp}
% \end{frame}

% \begin{frame}[fragile]{Задача 4}{Преоптоварување на оператори
% \texttt{+=, -=, *=, /=}}
% \lstinputlisting[firstline=22,lastline=43]{src/av3/z4.cpp}
% \end{frame}
% 
% \begin{frame}[fragile]{Задача 4}{Преоптоварување на оператори
% \texttt{+, -, *, /}}
% \lstinputlisting[firstline=61,lastline=76]{src/av3/z4.cpp}
% \end{frame}
% 
% \begin{frame}[fragile]{Задача 4}{Преоптоварување на оператори
% \texttt{<<, >>}}
% \lstinputlisting[firstline=44,lastline=59]{src/av3/z4.cpp}
% \end{frame}
% 
% \begin{frame}[fragile]{Задача 4}{Главна функција}
% \lstinputlisting[firstline=77]{src/av3/z4.cpp}
% \end{frame}
