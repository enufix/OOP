%%%%%%%%%%%%%%%%%%%%%%%%%%%%%%%%%%%%%%%%%
%%%%%%%%%% Content starts here %%%%%%%%%%
%%%%%%%%%%%%%%%%%%%%%%%%%%%%%%%%%%%%%%%%%
\section{Виртуелен деструктор}

\begin{frame}[fragile]{Пример 1}{Зошто е потребен виртуелен десктруктор?}
\lstinputlisting{src/av6/ex1.cpp}
\begin{tiny}
\begin{verbatim}
Konstruiram objekt od Osnovna
Konstruiram objekt od Izvedena
Unishtuvam objekt od Osnovna
\end{verbatim}
\end{tiny}
\end{frame}

\begin{frame}[fragile]{Пример 2}{Зошто е потребен виртуелен десктруктор?}
\lstinputlisting{src/av6/ex2.cpp}
\begin{tiny}
\begin{verbatim}
Konstruiram objekt od Osnovna
Konstruiram objekt od Izvedena
Unishtuvam objekt od Izvedena
Unishtuvam objekt od Osnovna
\end{verbatim}
\end{tiny}
\end{frame}

\begin{frame}{Задача 1}
\begin{scriptsize}
Да се креира хиерархија на класи за репрезентација на музичко и сликарско
уметничко дело. За потребите на оваа хиерархија да се дефинира полиморфична
класа \texttt{UmetnickoDelo} од која ќе бидат изведени двете класи
\texttt{MuzickoDelo} и \texttt{SlikarskoDelo}. 

Во класата \texttt{UmetnickoDelo} се чуваат податоци за годината кога е
изработено делото (int), авторот на уметничкото дело (динамички алоцирана низа
од знаци) и цената на уметничкото дело (float). За класата \texttt{MuzickoDelo}
дополнително се чува жанрот на делото (низа од 30 знаци). За класата
\texttt{SlikarskoDelo} дополнително се чуваат техниката во која е година е
изработено делото (низа од 30 знаци) и степенот на оштетеност на делото во
проценти (int).
\end{scriptsize}
\end{frame}

\section{Колоквиумски задачи}

\begin{frame}{Задача 1}
\begin{scriptsize}
За секој објект од двете изведени класи треба да бидат на располагање следниве
методи: 
\begin{itemize}
  \item Конструктор со aргументи кои одговараат на податочните членови
  \item set и get методи
  \item метода за пресметување на цената на уметничките дела \\
  Иницијалната цена на музичкото дело се зголемува за 10\% доколку тоа е
  изработено во 17 век.\\
  Иницијалната цена на сликарското дело процентуално се
намалува за степенот на неговата оштетеност
\item Преоптоварување на операторот $==$, кој ги споредува уметничките дела
според нивната цена
\item Преоптоварување на операторот $<<$ за печатење на сите податоци
за уметничките дела (за дома!) 
\end{itemize}
Сите променливи во класите се чуваат како приватни приватни.

\end{scriptsize}
\end{frame}


\begin{frame}[fragile]{Задача 1}{Решение 1/5}
\lstinputlisting[firstline=20,lastline=49]{src/av6/z1.cpp}
\end{frame}

\begin{frame}[fragile]{Задача 1}{Решение 2/5}
\lstinputlisting[firstline=50,lastline=76]{src/av6/z1.cpp}
\end{frame}
\begin{frame}[fragile]{Задача 1}{Решение 3/5}
\lstinputlisting[firstline=77,lastline=98]{src/av6/z1.cpp}
\end{frame}
\begin{frame}[fragile]{Задача 1}{Решение 4/5}
\lstinputlisting[firstline=99,lastline=129]{src/av6/z1.cpp}
\end{frame}
\begin{frame}[fragile]{Задача 1}{Решение 5/5}
\lstinputlisting[firstline=129]{src/av6/z1.cpp}
\end{frame}

\begin{frame}{Задача 2}
\begin{scriptsize}
Да се дефинира класа Casovnik, за која се чуваат информации за:
\begin{itemize}
  \item час (цел број),
  \item минути (цел број),
  \item секунди (цел број),
  \item призводител на часовникот (динамички алоцирана листа од знаци).
\end{itemize}
Од оваа класа да се изведат две нови класи \texttt{DigitalenCasovnik} и \texttt{AnalogenCasovnik}.
За дигиталниот часовник дополнително се чуваат информации за стотинките и
форматот на прикажување на времето (АМ или PM). За секоја од класите да се
дефинираат конструктори со аргументи. Во рамките на изведените класи да се
дефинира функција (Vreme) која го печати времето на дигиталниот часовник во
формат: производител, час, миннути, секунди, стотинки, АМ или PM. 
За аналогниот часовник функцијата печати: производител, час, минути, секунди.

Дополнително да се преоптовари операторот $==$ кој го споредува времето кое го
мерат два часовника и враќа \texttt{true} доколку времето на едниот часовник не
отстапува за повеќе од 30 секунди во однос на времето на другиот часовник, а во
спротивно враќа \texttt{false} (без да се води сметка за запоцнување на новиот
ден). 

Да се напише и надворешна функција (\texttt{Pecati}) која прима низа од
покажувачи кон класата \texttt{Casovnik} и нивниот број, а го печати времето на сите
часовници од низата.

\end{scriptsize}
\end{frame}

\begin{frame}[fragile]{Задача 2}{Решение 1/4}
\lstinputlisting[firstline=16,lastline=51]{src/av6/z2.cpp}
\end{frame}
\begin{frame}[fragile]{Задача 2}{Решение 2/4}
\lstinputlisting[firstline=52,lastline=75]{src/av6/z2.cpp}
\end{frame}
\begin{frame}[fragile]{Задача 2}{Решение 3/4}
\lstinputlisting[firstline=76,lastline=99]{src/av6/z2.cpp}
\end{frame}
\begin{frame}[fragile]{Задача 2}{Решение 4/4}
\lstinputlisting[firstline=101]{src/av6/z2.cpp}
\end{frame}


\begin{frame}{Задача 3}
\begin{scriptsize}
 Да се дефинира класа \texttt{Dogovor}, во која се чуваат информации за:
 \begin{itemize}
   \item број на договор (int),
   \item категорија на договор (низа од 50 знаци)
   \item динамички алоцирано поле од имињата на потпишувачите на договорот
   (имињата на потпишувачите не се подолги од 20 знаци)
   \item датум на потпишување на договорот (да се развие посебна класа за датуми
   во која ќе биде имплементиран операторот $<$ за споредување на два датуми).
 \end{itemize}
За потребите на оваа класа да се напишат преоптоварен конструктор со аргументите
на класата, set и get методи и операторот $<<$ за проследување на \texttt{ostream}
(печатење) на објект од класата \texttt{Dogovor}.

Дополнително да се креира класа \texttt{Klaster\_za\_dogovori} во која ќе се
чува динамички алоцирано поле од објекти од класата \texttt{Dogovor} и бројот на
објекти кои се чуваат во полето. 

За оваа класа да се преоптоварат: унарниот оператор $+=$ коj се однесува на
додавање на нов објект од класата \texttt{Dogovor} во рамките на полето
метода (\texttt{Potpisani\_dogovori}) на која се проследува објект од класата
\texttt{Datum} како параметар а таа враќа листа од потпишаните договори на проследениот
датум, операторот $<<$ за проследување на ostream (печатење) на \texttt{Dogovor}
објектите меѓусебно одвоени со нов ред.

\end{scriptsize}
\end{frame}

\begin{frame}[fragile]{Задача 3}{Решение 1/8}
\lstinputlisting[firstline=23,lastline=50]{src/av6/z3.cpp}
\end{frame}
\begin{frame}[fragile]{Задача 3}{Решение 2/8}
\lstinputlisting[firstline=51,lastline=76]{src/av6/z3.cpp}
\end{frame}
\begin{frame}[fragile,shrink=10]{Задача 3}{Решение 3/8}
\lstinputlisting[firstline=78,lastline=116]{src/av6/z3.cpp}
\end{frame}
\begin{frame}[fragile,shrink=10]{Задача 3}{Решение 4/8}
\lstinputlisting[firstline=78,lastline=116]{src/av6/z3.cpp}
\end{frame}
\begin{frame}[fragile,shrink=10]{Задача 3}{Решение 5/8}
\lstinputlisting[firstline=117,lastline=154]{src/av6/z3.cpp}
\end{frame}
\begin{frame}[fragile,shrink=10]{Задача 3}{Решение 6/8}
\lstinputlisting[firstline=156,lastline=192]{src/av6/z3.cpp}
\end{frame}
\begin{frame}[fragile,shrink=10]{Задача 3}{Решение 7/8}
\lstinputlisting[firstline=194,lastline=226]{src/av6/z3.cpp}
\end{frame}
\begin{frame}[fragile]{Задача 3}{Решение 8/8}
\lstinputlisting[firstline=228]{src/av6/z3.cpp}
\end{frame}


\begin{frame}{Задача 4}
\begin{scriptsize}
Да се дефинира класа \texttt{Otpornik} во која се чува вредноста на импедансата на
отпорникот $(Z=R)$. Класата треба да има метод кој ќе ја пресметува напонот на
краевите на отпорникот кога низ него тече дадена струја $I (U=I*|Z|)$. Од
класата отпорник да се изведе класата  \texttt{Impedansa} која ќе работи со комплексни импеданси
$(Z=R+jX)$. Класите треба да овозможуваат собирање на вредностите на две
импеданси преку операторот $+$ (сериска врска на две компоненети). Дополнително
да се преоптовари и операторот $<<$.

\end{scriptsize}
\end{frame}

\begin{frame}[fragile]{Задача 4}{Решение 1/3}
\lstinputlisting[firstline=8,lastline=35]{src/av6/z4.cpp}
\end{frame}
\begin{frame}[fragile]{Задача 4}{Решение 2/3}
\lstinputlisting[firstline=37,lastline=60]{src/av6/z4.cpp}
\end{frame}
\begin{frame}[fragile,shrink=10]{Задача 4}{Решение 3/3}
\lstinputlisting[firstline=62]{src/av6/z4.cpp}
\end{frame}


\begin{frame}{Задача 5}
\begin{scriptsize}
Да се развие класа \texttt{Poligon} која ќе претставува полигонална дво-дименнзионална
слика  во правоаголен кооординатен систем. Сликата е претставена како множество
од темиња (точки) на полигонот (динамички алоцирана листа). 

Класата треба да
овозможува поместување на фигурата по двете оски одеднаш како и пресметка на
периметарот на сликата. 

Да се преоптоварат релационите оператори $==$, $!=$, $<$ и $>=$
кои ќе споредуваат два полигони според вредностите на периметарот.

\end{scriptsize}
\end{frame}

\begin{frame}[fragile]{Задача 5}{Решение 1/4}
\lstinputlisting[firstline=8,lastline=26]{src/av6/z5.cpp}
\end{frame}
\begin{frame}[fragile,shrink=10]{Задача 5}{Решение 2/4}
\lstinputlisting[firstline=27,lastline=59]{src/av6/z5.cpp}
\end{frame}
\begin{frame}[fragile,shrink=10]{Задача 5}{Решение 3/4}
\lstinputlisting[firstline=61,lastline=93]{src/av6/z5.cpp}
\end{frame}
\begin{frame}[fragile]{Задача 5}{Решение 4/4}
\lstinputlisting[firstline=95]{src/av6/z5.cpp}
\end{frame}