%%%%%%%%%%%%%%%%%%%%%%%%%%%%%%%%%%%%%%%%%
%%%%%%%%%% Content starts here %%%%%%%%%%
%%%%%%%%%%%%%%%%%%%%%%%%%%%%%%%%%%%%%%%%%
\section{Inheritance}

\begin{frame}{Problem 1}
\begin{scriptsize}
Define a class for representing publication. For each publication there is
information for year of publishing (integer) and for the name of publishing
house (char[100]).

From this class derive (public) a new class for book. For each book additionally
there is number of pages.

Also derive (protected) a new class for newspaper. For each newspaper there is a
serial number.

From class newspaper derive (private) class for daily newspaper. For each daily
newspaper there is information for the day and month of publishing.

Test the access to the data with various functions!

\end{scriptsize}

\end{frame}

\begin{frame}[fragile]{Problem 1}{Solution 1/5}
\lstinputlisting[lastline=24]{src/av7/p1.cpp}
\end{frame}

\begin{frame}[fragile]{Problem 1}{Solution 2/5}
\lstinputlisting[firstline=25,lastline=42]{src/av7/p1.cpp}
\end{frame}

\begin{frame}[fragile]{Problem 1}{Solution 3/5}
\lstinputlisting[firstline=44,lastline=62]{src/av7/p1.cpp}
\end{frame}

\begin{frame}[fragile]{Problem 1}{Solution 4/5}
\lstinputlisting[firstline=63,lastline=75]{src/av7/p1.cpp}
\end{frame}

\begin{frame}[fragile]{Problem 1}{Solution 5/5}
\lstinputlisting[firstline=76]{src/av7/p1.cpp}
\end{frame}

\begin{frame}{Problem 2}
\begin{scriptsize}
Define a class for hotel reservation. For each hotel reservation is stored the
number of days, number of persons, name for contact. The price for the
reservation is 25 EUR per day per person.

Define a function \texttt{getPrice()} that returns the price of the reservation.
Define a function \texttt{getPrice(int advance)} that returns the price if the
user pays advance.

Derive a new class breakfast hotel reservation for reserving hotel room with
breakfast. The price for breakfast for one person is 5 EUR per day. Override the
respective function \texttt{getPrice(int advance)}. 

\end{scriptsize}
 
\end{frame}

\begin{frame}[fragile]{Problem 2}{Solution 1/4}
\lstinputlisting[lastline=29]{src/av7/p2.cpp}
\end{frame}

\begin{frame}[fragile]{Problem 2}{Solution 2/4}
\lstinputlisting[firstline=31,lastline=46]{src/av7/p2.cpp}
\end{frame}

\begin{frame}[fragile]{Problem 2 ++}
\begin{scriptsize}
Define a class Hotel with info about the name and the balance. In the class
define a function:

\texttt{int reserve(HotelReservation &hr, int advance);}

With this function we should pay for hotel reservation. If the payment exceeds
the needed amount, the function should return the change. The payment should add
to the balance of the hotel.

What will happen if the argument of HotelReservation is not reference?

\end{scriptsize}
\end{frame}


\begin{frame}[fragile]{Problem 2}{Solution 3/4}
\lstinputlisting[firstline=48,lastline=67]{src/av7/p2.cpp}
\end{frame}

\begin{frame}[fragile]{Problem 2}{Solution 4/4}
\lstinputlisting[firstline=69]{src/av7/p2.cpp}
\end{frame}

\begin{frame}{Problem 3}
\begin{scriptsize}
Define abstract class for representing a geometric shape. Each geometric shape
has height and base which can be different for each shape.

Add the following functions in the class:

\begin{itemize}
  \item \texttt{print()} print the shape info
  \item \texttt{volume()} the volume of the shape
  \item \texttt{getHeight()} returns the height of the shape
\end{itemize}

From this class derive classes for cylinder, cone and cuboid. For the cylinder
and cone, the information for the radius of the basis is stored and for the
cuboid, lenght of sides a and b.
\end{scriptsize}
 
\end{frame}

\begin{frame}[fragile]{Problem 3}{Solution 1/4}
\lstinputlisting[firstline=48,lastline=67]{src/av7/p3.cpp}
\end{frame}

\begin{frame}{Problem 3 ++}
In \texttt{main} declare and initialize dynamically array of pointers to the
class gemotric shape. From this array:

\begin{enumerate}
  \item Find the shape with largest volume, using the function: \texttt{void
  largestShape(GemotricShape **array, int n)}
  \item Find the number of geometric shapes that doesn't have basis circle using
  the function: \texttt{double getRadius(GeometricShape *g);} this function
  returns the radius of the basis of the shapes that have circle basis, and -1
  otherwise.
\end{enumerate}

\end{frame}

