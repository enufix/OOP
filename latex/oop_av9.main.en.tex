%%%%%%%%%%%%%%%%%%%%%%%%%%%%%%%%%%%%%%%%%
%%%%%%%%%% Content starts here %%%%%%%%%%
%%%%%%%%%%%%%%%%%%%%%%%%%%%%%%%%%%%%%%%%%

\begin{frame}{Problem 1}{Part 1}
Model a base class \texttt{PaymentCard}, and two derived classes
\texttt{CreaditCard} and \texttt{DebitCard}. Each card is described with its id
number and the current balance.

When paying with debit cards, on each amount there is a discount of 5\% for ALL
users and this percent cannot be changed.

When paying with credit cards, on each amount there is discount of 10\% if the
limit is over 6,000, and 3\% otherwise. The 10\% limit is the same for all users
but it can be changed from state bank.

\end{frame}

\begin{frame}[fragile]{Problem 1}{Solution 1/5}
\lstinputlisting[lastline=26]{src/av9/p91.cpp}
\end{frame}

\begin{frame}[fragile]{Problem 1}{Solution 2/5}
\lstinputlisting[firstline=28,lastline=45]{src/av9/p91.cpp}
\end{frame}

\begin{frame}[fragile]{Problem 1}{Solution 3/5}
\lstinputlisting[firstline=47,lastline=66]{src/av9/p91.cpp}
\end{frame}

\begin{frame}{Problem 1}{Part 2}

Create a class \texttt{CashRegister} where customers can pay with cards or in
cash. For each cash register there are two amounts:

\begin{itemize}
  \item amount payed in cash
  \item amount payed with cards
\end{itemize}

and each object of this class has the date when it is created.

Implement the following two functions in this class:

\begin{itemize}
  \item \texttt{pay(double)} - for paying in cash
  \item \texttt{pay(double, PaymentCard)} - for paying with card
\end{itemize}

\end{frame}

\begin{frame}[fragile]{Problem 1}{Solution 4/5}
\lstinputlisting[firstline=68,lastline=96]{src/av9/p91.cpp}
\end{frame}

\begin{frame}[fragile]{Problem 1}{Solution 5/5}
\lstinputlisting[firstline=98]{src/av9/p91.cpp}
\end{frame}


\begin{frame}{Problem 2}
\begin{scriptsize}
Part from the products from one store after the new policy of the store must
have some discount. To acomplish this the store system must model abstract class
Discount. This class keeps info about the exchange rates for euros and dollars
in denars and each class deriving from it must implement the following methods:

\begin{itemize}
  \item \texttt{float discount\_price()}
  \item  \texttt{float price()}
  \item \texttt{void print\_rule()}
\end{itemize}

For each Product the store keeps info about the name and the price. Products are
divided in few types: FoodProduct, Drinks and Cosmetics. According to the new
policy, food does not have discount. Alchohol drinks more expensive than 20
euros have discount of 5\%, and alchohol free drinks from the brand CocaCola
have discount of 10\%. All cosmetic products have discount of 12\%, and those
more expensive than 20 dollars have discount of 14\%.

\end{scriptsize}
 
\end{frame}

\begin{frame}{Problem 2}{Part 2}
\begin{scriptsize}
Compute the total price of all products with the discount applied.

The price of the products should not be negative. Throw an exception in the
constructor of the class Product, if the data is not ok.

Try different strategies of handling the exception:
\begin{itemize}
  \item Catch the exception in the constructor, so the price is set to 0
  \item Catch the exception in the main function, where we instantiate objects
  of classes derived from Product.
\end{itemize}

\end{scriptsize}
\end{frame}

\begin{frame}[fragile,shrink=0.9]{Problem 2}{Solution 1/5}
\lstinputlisting[lastline=36]{src/av9/p92.cpp}
\end{frame}

\begin{frame}[fragile]{Problem 2}{Solution 2/5}
\lstinputlisting[firstline=37,lastline=59]{src/av9/p92.cpp}
\end{frame}

\begin{frame}[fragile]{Problem 2}{Solution 3/5}
\lstinputlisting[firstline=61,lastline=78]{src/av9/p92.cpp}
\end{frame}

\begin{frame}[fragile]{Problem 2}{Solution 4/5}
\lstinputlisting[firstline=80,lastline=104]{src/av9/p92.cpp}
\end{frame}

\begin{frame}[fragile,shrink=0.9]{Problem 2}{Solution 5/5}
\lstinputlisting[firstline=106]{src/av9/p92.cpp}
\end{frame}