%%%%%%%%%%%%%%%%%%%%%%%%%%%%%%%%%%%%%%%%%
%%%%%%%%%% Content starts here %%%%%%%%%%
%%%%%%%%%%%%%%%%%%%%%%%%%%%%%%%%%%%%%%%%%

\begin{frame}[fragile]{Virtual destructor}{When and why we need virtual
destructor?}
\lstinputlisting{src/av11/e111.cpp}
\end{frame}

\begin{frame}[fragile]{Virtual destructor}{When and why we need virtual
destructor?}
\lstinputlisting{src/av11/e112.cpp}
\end{frame}

\begin{frame}{Problem 1}{1/2}
\begin{scriptsize}
Create a hierarchy of classes that represent music and painting work. 
For each art work we have:
\begin{itemize}
  \item year (int)
  \item author (dynamically allocated char array)
  \item price (float) - should not be negative (Exception)
\end{itemize}
For each music work we have additionally the genre (char array), and for each
painting work we have additionally the technique (char array) and percent of
damage (int).

\end{scriptsize}
\end{frame}

\begin{frame}{Problem 1}{2/2}
\begin{scriptsize}
For each object of the classes should be implemented the following methods:
\begin{itemize}
  \item \texttt{float price()}
  \begin{itemize}
    \begin{scriptsize}
    \item the initial price of the music work is increased for x\% if it dates
    before the 17th century. The value x can be changed but it's save for all
    objects.
    \item the initial price of the painting work is decreased according to the
    damage percent.
    \end{scriptsize}
   \end{itemize}
   \item operator $>$ - compares by their price
   \item opearotr $<<$ - outputs the author, year and price
\end{itemize}

\end{scriptsize}
\end{frame}

\begin{frame}[fragile]{Problem 1}{Solution 1/3}
\lstinputlisting[lastline=26]{src/av11/p111.cpp}
\end{frame}

\begin{frame}[fragile]{Problem 1}{Solution 2/3}
\lstinputlisting[firstline=28,lastline=54]{src/av11/p111.cpp}
\end{frame}

\begin{frame}[fragile]{Problem 1}{Solution 3/5}
\lstinputlisting[firstline=56, lastline=70]{src/av11/p111.cpp}
\end{frame}

\begin{frame}[fragile]{Problem 1}{Solution 4/5}
\lstinputlisting[firstline=72, lastline=88]{src/av11/p111.cpp}
\end{frame}

\begin{frame}[fragile]{Problem 1}{Solution 5/5}
\lstinputlisting[firstline=90, lastline=112]{src/av11/p111.cpp}
\end{frame}

\begin{frame}{Problem 2}{Function template}
\begin{scriptsize}
Write a function templates:
\begin{itemize}
  \item that will merge two arrays of same type
  \item that will print elements from array
  \item that will create a new array which is subarray from the original,
  defined with start and end.
\end{itemize}
\end{scriptsize}
\end{frame}

\begin{frame}[fragile]{Problem 2}{Solution 1/2}
\lstinputlisting[firstline=1, lastline=28]{src/av11/p112.cpp}
\end{frame}

\begin{frame}[fragile]{Problem 2}{Solution 2/2}
\lstinputlisting[firstline=30]{src/av11/p112.cpp}
\end{frame}

\begin{frame}{Problem 3}
\begin{scriptsize}
Model a class Bicycle. For each bicycle we have model name, mass and diameter of
the wheel (inches).

In bicycle competitions there are three types of races: road races, mountain
races and hybrid races. Dependant of the race type there are also three types of
bicycles. Bicycles driven on road have extra info about the width of the tyres,
mountain bicycles have number of suspension systems. Create a class hierarchy.
\end{scriptsize}
\end{frame}

\begin{frame}{Problem 3}{Exceptions}
\begin{itemize}
  \item diameter of the wheel is in range (15 - 29)
  \item when user tries to create object with invalid value, an object with
  fixed diameter 21 should be created.
\end{itemize}
\end{frame}

\begin{frame}{Problem 3}{Polymorphism}
In each class implment a function \texttt{float getRangCoefficient()} that
returns the coefficient of ranging the bicycle.

\begin{itemize}
  \item for road bicycles $(2.5 * width) * 2 + diameter * 0.2$
  \item for mountain bicycles $number\_of\_suspensions + (27 - diameter) * 0.8$
  \item for hybrid is the minimum of the road and mountain bicycles
\end{itemize}
\end{frame}

\begin{frame}{Problem 3}{Static function}
In the class bicycle define a field with MAX allowed mass for competition. This
information is defined by the bicycle federation and it's same for all bicycles.
There should be a function that will check if bicycle can compete with
definition:

\texttt{bool canCompete(Bicycle \&b)}
\end{frame}

\begin{frame}{Problem 3}{Composition}
Define a class \texttt{Competitor} that has info for name and pointer to
dynamically allocated bicycle he/she uses in competitions.
\end{frame}

\begin{frame}{Problem 3}{Global function}
Define a global function for printing info of competitors by categories and
sorted in increasing order by their coeffitient. This defines their starting
position. For each competitor print the starting position, name and model of
bicycle and rang coeffitient.
\end{frame}

\begin{frame}{Problem 3}{Main function}
In the main function read data for all competitors of a bicycle race in Skopje.
Then print the starting positions of the competitors (only those who can
compete) with a maximum allowed mass of 15kg for bicycle.
\end{frame}

