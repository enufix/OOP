%%%%%%%%%%%%%%%%%%%%%%%%%%%%%%%%%%%%%%%%%
%%%%%%%%%% Content starts here %%%%%%%%%%
%%%%%%%%%%%%%%%%%%%%%%%%%%%%%%%%%%%%%%%%%
\section{Наследување}

\begin{frame}{Пример 1/4}{\texttt{protected} наследување и \texttt{protected} членови на
класата}
\lstinputlisting[lastline=24]{src/av5/ex1.cpp}
\end{frame}

\begin{frame}[fragile]{Пример 2/4}{\texttt{protected} наследување и \texttt{protected} членови на
класата}
\lstinputlisting[firstline=25,lastline=39]{src/av5/ex1.cpp}
\end{frame}

\begin{frame}[fragile]{Пример 3/4}{\texttt{protected} наследување и
\texttt{protected} членови на класата}
\lstinputlisting[firstline=40,lastline=58]{src/av5/ex1.cpp}
\end{frame}

\begin{frame}[fragile]{Пример 4/4}{\texttt{protected} наследување и
\texttt{protected} членови на класата}
\lstinputlisting[firstline=59]{src/av5/ex1.cpp}
\begin{tikzpicture}[overlay,remember picture]
        \pgftransformshift{\pgfpointanchor{current page}{center}}
        \node[
            ellipse callout,
            draw=red,
            thick,
            fill=yellow,
            decorate,
            callout relative pointer=(195:3.5cm),
            text width=0.15\textwidth,
            align=center,
            anchor=center
            ] at (3.5,0) {Грешка};
\end{tikzpicture}
\begin{tikzpicture}[overlay,remember picture]
        \pgftransformshift{\pgfpointanchor{current page}{center}}
        \node[
            ellipse callout,
            draw=red,
            thick,
            fill=yellow,
            decorate,
            callout relative pointer=(180:3cm),
            text width=0.15\textwidth,
            align=center,
            anchor=center
            ] at (3.5,-2) {Грешка};
    \end{tikzpicture}
\end{frame}

\begin{frame}{Задача 1}
Да се напише класа за работа со функција \texttt{f(x, y)} претставена како множество од
точки во 3Д простор (динамички алоцирана меморија за објекти од класата \texttt{Tocka3D}
која исто така треба да се имплементира). 

Функцијата f треба да  ги поддржува оперторите \texttt{+=} за додавање нова точка во
множеството точки, \texttt{<<}  за печатење на функцијата \texttt{f} и \texttt{[i]}  за промена  на 
точката  на  позиција \texttt{i}. Исто така, класата треба  да  има  метод 
кој ќе врши интерполација на точките од функцијата и ќе ја враќа новодобиената  
интерполирана функција.

\end{frame}

\begin{frame}[fragile]{Задача 1}{Решение 1/5}
\lstinputlisting[lastline=28]{src/av5/z1.cpp}
\end{frame}

\begin{frame}[fragile]{Задача 1}{Решение 2/5}
\lstinputlisting[firstline=29,lastline=50]{src/av5/z1.cpp}
\end{frame}

\begin{frame}[fragile]{Задача 1}{Решение 3/5}
\lstinputlisting[firstline=51,lastline=79]{src/av5/z1.cpp}
\end{frame}

\begin{frame}[fragile,shrink=10]{Задача 1}{Решение 4/5}
\lstinputlisting[firstline=80,lastline=113]{src/av5/z1.cpp}
\end{frame}

\begin{frame}[fragile]{Задача 1}{Решение 5/5}
\lstinputlisting[firstline=114]{src/av5/z1.cpp}
\end{frame}

\begin{frame}{Задача 2}{}
 Да се развие класа за работа со производи со можност за менување на цената и
 пресметка на заштедата при можен избор од два производи. Да се преоптовари
 операторот \texttt{<} за споредба на  продажната цена на два производи. Од оваа класа да
 се изведе класа за работа со производи на процентуален попуст за кои цената
 која се наплаќа при купување се намалува за соодветниот попуст. Да се напише
 функција која на влез ќе прима два производи (од ист или различен вид) и ќе го
 советува купувачот кој е поефтин и колкава е заштедата.
\end{frame}

\begin{frame}[fragile]{Задача 2}{Решение 1/4}
\lstinputlisting[lastline=28]{src/av5/z2.cpp}
\end{frame}

\begin{frame}[fragile]{Задача 2}{Решение 2/4}
\lstinputlisting[firstline=29,lastline=48]{src/av5/z2.cpp}
\end{frame}

\begin{frame}[fragile]{Задача 2}{Решение 3/4}
\lstinputlisting[firstline=49,lastline=73]{src/av5/z2.cpp}
\end{frame}

\begin{frame}[fragile]{Задача 2}{Решение 4/4}
\lstinputlisting[firstline=74]{src/av5/z2.cpp}
\end{frame}

\begin{frame}{Задача 3}
    Да  се  состави хиерархија на  класи  за  конверзии од  еден  систем на 
    единици  во  друг. Основната класа чува две променливи, \texttt{val1} и \texttt{val2}, кои се
    однесуваат на иницијалната и конвертираната вредност. Да се имплементираат
    функции \texttt{getinit()} and \texttt{getconv()}, кои ја враќаат  иницијалната  односно 
    конвертираната  вредност  соодветно. Функцијата која ја прави самата
    конверзија, \texttt{compute()}, е чиста виртуелна функција. Да се  додадат две
    изведени класи за конверзија на литри во галони и фаренхајти во целзиусови
    степени.
\end{frame}


\begin{frame}[fragile]{Задача 3}{Решение 1/3}
\lstinputlisting[lastline=20]{src/av5/z3.cpp}
\end{frame}

\begin{frame}[fragile]{Задача 3}{Решение 2/3}
\lstinputlisting[firstline=21,lastline=50]{src/av5/z3.cpp}
\end{frame}

\begin{frame}[fragile]{Задача 3}{Решение 3/3}
\lstinputlisting[firstline=51]{src/av5/z3.cpp}
\end{frame}

\begin{frame}{Задача 4}
\begin{scriptsize}
    Да се напише класа \texttt{Telefon} со која ќе се овозможи работа со податоци за
    телефони. За секој телефон се чуваат година на производство (int), почетна
    цена (int) и модел на производот (низа од 40 знаци). 
    
    Од класата \texttt{Telefon} да се изведе класа \texttt{Mobilen} за кој дополнително се чуваат
    информации за ширината (float) и висината (float), дебелина (float) во центиметри. 
    
    Од класата \texttt{Telefon} да се изведе и класа \texttt{Fiksen} за кој дополнително ќе се
    чува информација за неговата тежина во грамови (int). За секоја од класите да се
    напише соодветен конструктор, set и get методи и да се преоптовари
    операторот \texttt{<<} за печатење. 
    
    За секоја од изведените класи да се обезбеди функција \texttt{presmetajVrednost()} за
    пресметка на моменталната цена како: 
    \begin{itemize}
      \item За мобилен телефон: 0.95\% од цената ако e произведен пред 2010
      година и цената зголемена за 5\% ако е потенок од 0.7cm
     \item За фиксен телефон: 0.98\% од цената ако e произведен пред 2009
     година и цената зголемена за 7\% ако е полесен од 100гр
    \end{itemize}
     
    Да се напише функција која на влез прима низа од покажувачи кон класата
    \texttt{Telefon} и ја печати цената на телефонот со најмала цена. Да се напише \texttt{main}
    функција во која ќе се тестираат имплементираните функции во класите.
\end{scriptsize}
\end{frame}

\begin{frame}[fragile]{Задача 4}{Решение 1/4}
\lstinputlisting[lastline=29]{src/av5/z4.cpp}
\end{frame}

\begin{frame}[fragile]{Задача 4}{Решение 2/4}
\lstinputlisting[firstline=30,lastline=50]{src/av5/z4.cpp}
\end{frame}

\begin{frame}[fragile]{Задача 4}{Решение 3/4}
\lstinputlisting[firstline=51,lastline=69]{src/av5/z4.cpp}
\end{frame}

\begin{frame}[fragile]{Задача 4}{Решение 4/4}
\lstinputlisting[firstline=70]{src/av5/z4.cpp}
\end{frame}

\begin{frame}{Задача 5}
\begin{scriptsize}
Компанијата ФИНКИ-Фото нуди фотоапарати од различни производители и со различни
карактеристики. За потребите на компанијата треба да се развијат класи кои ги
опишуваат фотоапаратите кои таа ги нуди. За таа цел, треба да се развие класа
\texttt{Fotoaparat}, во која се чуваат информации за:
\begin{itemize}
  \item модел (низа од 30 знаци),
  \item основна цена (реален број),
  \item резолуција изразена во мегапиксели (цел број).
\end{itemize}

ФИНКИ-Фото располага со два вида фотоапарати за кои треба да се изведат две
посебни класи \texttt{DSLR} фотоапарати и \texttt{Kompaktni} фотоапарати. За
\texttt{DSLR} фотоапаратите се чува дополнителна информација за видот на
објективот (низа од 20 знаци) и цена на објективот (реален број), а за
компактните фотоапарати дополнително се чува информација за видот на зумот
(boolean променлива – вредност true ако станува збор за оптички зум, вредност
false – за дигитален зум). 

За секоја од дефинираните класи да се напише конструктор кој ќе ги иницијализира
сите атрибути за апаратите и да се преоптовари операторот $<<$ за печатење.

\end{scriptsize}
\end{frame}

\begin{frame}{Задача 5}
\begin{scriptsize}
За секоја од класите \texttt{DSLR} и \texttt{Kompaktni} треба да се имплементира
метод за пресметување на цената на фотоапаратите според следниот критериум: 
\begin{itemize}
  \item За DSLR фотоапарати:
  \begin{itemize}
  \item Основната цена се зголемува за 15\% ако имаат резолуција не помала од 15
  мегапиксели
  \item потоа се додава цената на објективот.
  \end{itemize}
  \item За компактните фотоапарати: 
  \begin{itemize}
    \item Основната цена се зголемува за 12\% ако имаат резолуција не помала од
    10 мегапиксели
    \item Потоа се пресметува покачување на цената уште за 10\% ако станува збор
    за оптички зум
  \end{itemize}  
\end{itemize}
Да се преоптовари операторот $<$ за споредба на фотоапарати, кои може да бидат
од различен вид, според нивната цена.
Да се напише и фунција (\texttt{najmalaCena}) која како аргумент прима низа од
покажувачи кон разни видови фотоапарати, како и нивниот број, а ја печати цената
на фотоапаратот со најмала цена.
\end{scriptsize}
\end{frame}

\begin{frame}[fragile]{Задача 5}{Решение 1/3}
\lstinputlisting[firstline=5,lastline=33]{src/av5/z5.cpp}
\end{frame}

\begin{frame}[fragile]{Задача 5}{Решение 2/3}
\lstinputlisting[firstline=34,lastline=57]{src/av5/z5.cpp}
\end{frame}

\begin{frame}[fragile]{Задача 5}{Решение 3/3}
\lstinputlisting[firstline=58,lastline=82]{src/av5/z5.cpp}
\end{frame}


