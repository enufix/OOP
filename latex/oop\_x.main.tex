%%%%%%%%%%%%%%%%%%%%%%%%%%%%%%%%%%%%%%%%%
%%%%%%%%%% Content starts here %%%%%%%%%%
%%%%%%%%%%%%%%%%%%%%%%%%%%%%%%%%%%%%%%%%%

\begin{frame}{Интервју за работа}{Дали сакате да работите во некоја од следните
компании?}
\includegraphics[width=3cm]{images/google}\hfill
\includegraphics[width=3cm]{images/microsoft}\\
\includegraphics[width=3cm]{images/amazon}\hfill
\includegraphics[width=3cm]{images/apple}
\end{frame}

\begin{frame}{Прашање на интервју}{The classic "eval"
interview question}
\begin{itemize}
  \item Прв пат поставено на интервју во Амазон
  \item Прашањето е доста обемно и содржи многу важни вештини со кои треба да ги
  поседува еден квалитетен софтверски инженер:
  \begin{itemize}
  \item \textbf{ООП дизајн}
  \item \textbf{рекурзија}
  \item бинарни дрва
  \item \textbf{полиморфизам}
  \item \ldots
  \end{itemize}
\end{itemize}
\end{frame}

\begin{frame}{Прашањето}{1/2}
Во одреден момент, кандидатот конечно сфаќа дека секоја аритметичка операција
може да се претстави како бинарно дрво, ако препоставиме дека ги користиме само
основните бинарни оператори како $+, -, *, /$. Листовите се броеви, а
сите внатрешни јазли се оператори. Евалуацијата на изразот значи изминување на ова
дрво. 

Не го разбирате ова? Не е важно\ldots

Ова е сеуште интересен проблем.
\end{frame}

\begin{frame}{Прашањето}{2/2}
Првиот дел од прашањето е како од стринг кој претставува некаков аритметички
израз (пр. ``2 + (2)'') го трансформираме во дрво за изразот.

Вториот дел е: да речеме дека ова е проект за двајца луѓе и вашиот партнер е
задолжен за трансформацијата на изразот во дрво, а на вас останува лесниот дел.
Треба да напишете соодветни класи кои ќе ги употреби вашиот партнер за да ја
заврши неговата задача.

\end{frame}

\begin{frame}{Одговорот}{Најдоброто решение користи полиморфизам!}
Да се обидеме да одговориме на следните прашања?
\begin{itemize}
  \item Кои се заедничките карактеристики или однесување на различни изрази?
  \item Кои се разликите?
  \item Кои методи/полиња треба да се наследат, а кои да се имплементираат
  посебно?
\end{itemize}

\end{frame}


\begin{frame}[fragile]{Expression interface}{Решение 1/4}
\lstinputlisting{src/avx/Expression.java}
\end{frame}

\begin{frame}[fragile]{BinaryExpression abstract class}{Решение 2/4}
\lstinputlisting{src/avx/BinaryExpression.java}
\end{frame}

\begin{frame}[fragile]{ValueNode class}{Решение 3/4}
\lstinputlisting{src/avx/ValueNode.java}
\end{frame}

\begin{frame}[fragile]{PlusExpression class}{Решение 4/4}
\lstinputlisting{src/avx/PlusExpression.java}
\end{frame}

\begin{frame}[fragile,shrink=40]{Демо}
\vspace{2cm}
\begin{lstlisting}
public class Demo {
    public static void main(String[] args) {
        Expression expression = new ValueNode(2); // 2
        processExpression(expression);
        expression = new PlusExpression(new ValueNode(2), new ValueNode(3)); // (2 * 3)
        processExpression(expression);
        Expression mulExpression = new MultiplyExpression(new ValueNode(2), new ValueNode(3));
        Expression divExpression = new DivideExpression(new ValueNode(10), new ValueNode(2));
        expression = new PlusExpression(mulExpression, divExpression); // (2 * 3)
        processExpression(expression); // (2 * 3) + (10 / 2)
    }
    
    public static void processExpression(Expression expression) {
        System.out.println(expression.expression());
        System.out.println(String.format("Result: %.2f", expression.eval()));
    }
}
\end{lstlisting}
\end{frame}

